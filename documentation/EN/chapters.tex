%% ==============================
\chapter{Single-bidder and Single-item Randomization Auction}
\label{ch:body of thesis}
%% ==============================
In this chapter, we evaluate APX with different probability distribution under single-bidder and single-item randomization auction experiment. We still use \textit{take-it-or-leave-it} auction mechanism. The robust paper defines a specific \textit{randomized} selling mechanism, which essentially corresponds to the lottery proposed by Carrasco et al. [paper reference]
\begin{definition}[Log-Lottery]
	\label{definition:loglottery}
	Fix any $\mu > 0$ and $\sigma \geqslant 0$. A log-lottery is a randomized mechanism that sells at a price $P^{log}_{\mu,\sigma}$, which is distributed over the nonnegative interval support $[\pi_1, \pi_2]$ according to the cdf
	\begin{equation}\notag
		F^{log}_{\mu,\sigma} (x) =  \frac{\pi_2 \ln\frac{x}{\pi_1} -(x - \pi_1)}{\pi_2 \text{ln} \frac{\pi_2}{\pi_1} - (\pi_2-\pi_1)}
	\end{equation}
	where parameters $\pi_1,\pi_2$ are the (unique) solutions of the system
	\begin{equation}
		\begin{cases} \pi_1 (1 + \text{ln}\frac{\pi_2}{\pi_1}) = \mu   \\    \pi_1(2\pi_2 - \pi_1) = \mu^2 + \sigma^2  \end{cases}
	\end{equation}
\end{definition}
We will sometimes slightly abuse notation and use $P^{log}_{\mu,\sigma}$ to refer both the log-lottery mechanism and the corresponding variable of the prices.

This time, we sample $N$ $P^{log}_{\mu,\sigma}$s from log-lottery distribution (i.e. $N = 10000$), and for each $P^{log}_{\mu,\sigma}$ we perform our Auction Experiment Procedure \ref{alg:auctionexperiment} from Chapter 2. We just need replace $p_D$ with $P^{log}_{\mu,\sigma}$ in the procedure. 
\section{Log-Lottery Sampling}
The Log-Lottery distribution defined in Definition \ref{definition:loglottery} is not a regular distribution, therefore there is no sampling function in Python we can use directly. First we can check its pdf
\begin{equation}\notag
	f^{log}_{\mu,\sigma} (x) =  \frac{\pi_2\frac{1}{x} -1}{\pi_2 \text{ln} \frac{\pi_2}{\pi_1} - (\pi_2-\pi_1)}
\end{equation} 
$f^{log}_{\mu,\sigma} (x)$ is monotone decreasing on $[\pi_1,\pi_2]$, so $\text{max}(f^{log}_{\mu,\sigma} (x)) = f^{log}_{\mu,\sigma} (\pi_1)$.
Since this distribution is bounded on bounded support, we can use the rejection sampling technique. This rejection sampling requires a valid upper bound, usually also called envelop of the target density. For the Log-lottery, this upper bound is $f^{log}_{\mu,\sigma} (\pi_1)$, then we can propose an uniform density
\begin{center}
	$q(x) = \begin{cases} \frac{1}{\pi_2 - \pi_1}  & \text{if } x \in [\pi_1, \pi_2] \\ 0   & \text{otherwise } \end{cases}$
\end{center}
A constant $A = f^{log}_{\mu,\sigma} (\pi_1)\cdot (\pi_2 - \pi_1)$, then we know $A \cdot q(x) \geqslant	f^{log}_{\mu,\sigma} (x) , 	\forall x \in [\pi_1,\pi_2]$. Our rejection sampling steps can be described in Algorithm \ref{alg:boundRejection} below.
\begin{algorithm}
	\caption{Rejection Sampling Algorithm}\label{alg:boundRejection}
	\begin{algorithmic}[1]
		\Procedure{Rejection Sampling}{N}
		\State $n \gets 0$
		\State $x_{list}$ is empty
		\While{$n  \leqslant N$}\Comment{N is sample size}
			\State draw $x \sim q(x)$		
			\State compute acception probability $a := \frac{f^{log}_{\mu,\sigma} (x) }{Aq(x)}$
			\State draw a random sample $u \sim U[0,1]$
			\If{$u \leqslant a$}
				\State accept $x$, add it to $x_{list}$
				\State $n\gets n+1$	
			\EndIf
		\EndWhile
		\State \textbf{return} $x_{list}$
		\EndProcedure
	\end{algorithmic}
\end{algorithm}

\section{Uniform distribution}
Now our reserve price becomes $P^{log}_{\mu,\sigma}$ while other parameters remain the same. Given $U[a,b]$, under the log-lottery randomization selling mechanism, we can write down our expected revenue with reserve price $P^{log}_{\mu,\sigma}$  
\begin{equation}\notag
\begin{split}	
	\text{REV}(P^{log}_{\mu,\sigma} ;U[a,b]) &=\underset{ p \sim F^{log}_{\mu,\sigma} }{\mathbb{E}}\left[ p(1-F_{uniform}(p))\right] \\ &= \int_{\pi_1}^{\pi_2} p(1 -  \frac{p-a}{b-a})f^{log}_{\mu,\sigma}(p)dp\\ &=\int_{\pi_1}^{\pi_2} p(1 -  \frac{p-a}{b-a}) \cdot \frac{\pi_2\frac{1}{p} -1}{\pi_2 \text{ln}\frac{\pi_2}{\pi_1} - (\pi_2-\pi_1)} dp \\&=\int_{\pi_1}^{\pi_2} \frac{p(b-p)}{b-a} \cdot \frac{\pi_2 -p}{p(\pi_2 \text{ln}\frac{\pi_2}{\pi_1} - (\pi_2-\pi_1))} dp \\&=  \frac{1}{(b-a)(\pi_2 \text{ln}\frac{\pi_2}{\pi_1} - (\pi_2-\pi_1))}\int_{\pi_1}^{\pi_2} (b-p)(\pi_2-p)dp\\&=  \frac{1}{(b-a)\left(\pi_2 \text{ln}\frac{\pi_2}{\pi_1} - (\pi_2-\pi_1)\right)} \cdot \left( \frac{p^3}{3} -\frac{bp^2}{2} -\frac{\pi_2p^2}{2} + b\pi_2 p \Biggr|_{\pi_2}^{\pi_1} \right)              
\end{split}
\end{equation} 

From Chapter 2, we also know $\text{OPT}(U[a,b]) = \frac{b^{2}}{4(b-a)}$, then
\begin{equation}\notag
    \text{APX} = \frac{\text{OPT}(U[a,b])}{\text{REV}(P^{log}_{\mu,\sigma} ;U[a,b])} = \frac{b^2\left(\pi_2 \text{ln}\frac{\pi_2}{\pi_1} - (\pi_2-\pi_1)\right)}{4\left( \frac{p^3}{3} -\frac{bp^2}{2} -\frac{\pi_2p^2}{2} + b\pi_2 p \Biggr|_{\pi_2}^{\pi_1} \right)  }
\end{equation}
is a function depends on $b, \pi_1, \pi_2$. 

\section{Normal distribution}
Similarly the expected revenue of a truncated normal distribution $N_t(\mu, \sigma^2)$

\begin{equation}\notag
\begin{split}	
	\text{REV}(P^{log}_{\mu,\sigma} ;N_t(\mu, \sigma^2)) &=\underset{p \sim F^{log}_{\mu,\sigma} }{\mathbb{E}}\left[ p(1-F_{N_t}(p))\right] \\ &= \int_{\pi_1}^{\pi_2} p(1 -  k(F_{N}(p)-F_{N}(0)))f^{log}_{\mu,\sigma}(p)dp \\&=\int_{\pi_1}^{\pi_2} p(1 -  k(F_{N}(p)-F_{N}(0)))\cdot \frac{\pi_2 -p}{p(\pi_2 \text{ln}\frac{\pi_2}{\pi_1} - (\pi_2-\pi_1))} dp \\&=  \frac{1}{\pi_2 \text{ln}\frac{\pi_2}{\pi_1} - (\pi_2-\pi_1)}\int_{\pi_1}^{\pi_2} (1 -  k(F_{N}(p)-F_{N}(0)))(\pi_2-p)dp 
\end{split}
\end{equation} 


\section{Observation from experiment}

As we can see from the plots above for both uniform distribution and truncated normal distribution, we can see DAPX is always smaller than APX, which means for these two distribution and for one bidder and one items, running deterministic robust auction mechanism is better than running log lottery randomization.

However here also comes our questions: in general, randomization is better than deterministic mechanism. Here we compared one deterministic mechanism with one randomization mechanism given a certain valuation distribution. In the paper, APX is the ratio under worst case (worst probability distribution) with log lottery randomization. 

1.Randomization is better than deterministic, but given a valuation distribution, using one specific randomization may not be better. Thus whether log lottery is the best randomization distribution. How about using uniform or truncated normal distribution for price randomization? Especially is log lottery is the best for uniform and truncated normal distribution?
2.During the experiment, we use rejection sampling to generate random log lottery price, is the sample size sufficient enough?
3.A better deterministic mechanism can result a better approximate robust ratio than a specific randomization mechanism, i.e. better than log lottery randomization. However in worse case, APX is always better than DAPX. 






\chapter{Additional notes}



\section{truncated normal distribution part}

\begin{center}
	$f_t(x) = \begin{cases} k f(x) & \text{if } x \geqslant 0 \\ 0   & \text{otherwise } \end{cases}$
\end{center}

where $k$ is a normalizing constant. We can determine k by using the knowledge of the sum of the probablity equals to 1. Then we have: 

\begin{center}
	
	$1 = \int_{R}f_t(x) dx = \int_{0}^{\infty} k f(x)dx = k \cdot \int_{0}^{\infty} f(x) dx$
	
\end{center}

\begin{center}
	
	$\implies k = \frac{1}{\int_{0}^{\infty}f(x) dx } = \frac{1}{1 - F(0)}$
	
\end{center}

Here $x \geqslant 0 $, then the coresponding cdf is:
\begin{center}
	$F_t(x) = \int_{0}^{x} kf(t) dt =  k \cdot \int_{0}^{x} f(t) dt = k(F(x) - F(0))$
\end{center}






question to answer:\\
1.two random variables, X,Y, let Y = c*X where c>0 constant. let $F_X,f_X$ be the cdf, pdf of X what is the relation of $F_Y,f_y$ to $F_X,f_X$\\
$F_Y (y) = P(Y \geqslant y) = P(cX \leqslant y) = P(X \leqslant \frac{y}{c}) = F_X (\frac{y}{c})$\\
we know $f_Y(y) = f_X (g^{-1}(y))|\frac{d g^{-1}(y)}{dy}|$ thus:\\
$f_Y (y) = f_X(\frac{y}{c})\cdot|\frac{1}{c}| =\frac{1}{c}f_X(\frac{y}{c})$\\
2. what is the relation of myerson(Y) and myerson(X)\\
Myerson(X) optimal revenue can be achieved by a deterministic mechanism, let denot $v_x$ of optimal reverse price for X, then OPT(X) = $v_x(1-F_X(v_x))$, what if optimal reserve price for Y?
we can compute optimla reserve price by setting virtual valuation for Y equal to 0:
\begin{align}\notag
v - \frac{1-F_X (\frac{v}{c})}{\frac{1}{c}f_X(\frac{v}{c}) } &= 0 \\     
\notag  \frac{v}{c} - \frac{1-F_X (\frac{v}{c})}{ f_X(\frac{v}{c}) }  &= 0
\end{align}
set $v^{'} = \frac{v}{c}$, above equation has solution $v_x$.
Then denote optimal reserve price for Y: $v_y =c v_x$ then OPT(Y) = $v_y(1 - F_Y(v_y)) =cv_x(1 - F_X(\frac{cv_x}{c})) = cv_x(1 - F_X(v_x))$. Therefore OPT(Y) = cOPT(X)
3. what is the relation of REV(Y) and REV(x), given v as reserve price\\
REV(Y) = $v(1-F_Y(v)) = v(1-F_X(\frac{v}{c}))$\\
REV(X) =  $v(1-F_X(v))$ 
\begin{equation}\notag
\begin{split}	
v(1-F_X(\frac{v}{c})) &\overset{0 < c < 1}{ < }v(1-F_X(v))\\&\overset{c= 1}{ = } \\&\overset{c > 1}{ > }
\end{split}
\end{equation} 




Experiment Evaluation of Robust Revenue-Maximizing Auctions\\ \\


%%% Local Variables: 
%%% mode: latex
%%% TeX-master: "thesis"
%%% End: 